
% Default to the notebook output style

    


% Inherit from the specified cell style.




    
\documentclass[11pt]{article}

    
    
    \usepackage[T1]{fontenc}
    % Nicer default font (+ math font) than Computer Modern for most use cases
    \usepackage{mathpazo}

    % Basic figure setup, for now with no caption control since it's done
    % automatically by Pandoc (which extracts ![](path) syntax from Markdown).
    \usepackage{graphicx}
    % We will generate all images so they have a width \maxwidth. This means
    % that they will get their normal width if they fit onto the page, but
    % are scaled down if they would overflow the margins.
    \makeatletter
    \def\maxwidth{\ifdim\Gin@nat@width>\linewidth\linewidth
    \else\Gin@nat@width\fi}
    \makeatother
    \let\Oldincludegraphics\includegraphics
    % Set max figure width to be 80% of text width, for now hardcoded.
    \renewcommand{\includegraphics}[1]{\Oldincludegraphics[width=.8\maxwidth]{#1}}
    % Ensure that by default, figures have no caption (until we provide a
    % proper Figure object with a Caption API and a way to capture that
    % in the conversion process - todo).
    \usepackage{caption}
    \DeclareCaptionLabelFormat{nolabel}{}
    \captionsetup{labelformat=nolabel}

    \usepackage{adjustbox} % Used to constrain images to a maximum size 
    \usepackage{xcolor} % Allow colors to be defined
    \usepackage{enumerate} % Needed for markdown enumerations to work
    \usepackage{geometry} % Used to adjust the document margins
    \usepackage{amsmath} % Equations
    \usepackage{amssymb} % Equations
    \usepackage{textcomp} % defines textquotesingle
    % Hack from http://tex.stackexchange.com/a/47451/13684:
    \AtBeginDocument{%
        \def\PYZsq{\textquotesingle}% Upright quotes in Pygmentized code
    }
    \usepackage{upquote} % Upright quotes for verbatim code
    \usepackage{eurosym} % defines \euro
    \usepackage[mathletters]{ucs} % Extended unicode (utf-8) support
    \usepackage[utf8x]{inputenc} % Allow utf-8 characters in the tex document
    \usepackage{fancyvrb} % verbatim replacement that allows latex
    \usepackage{grffile} % extends the file name processing of package graphics 
                         % to support a larger range 
    % The hyperref package gives us a pdf with properly built
    % internal navigation ('pdf bookmarks' for the table of contents,
    % internal cross-reference links, web links for URLs, etc.)
    \usepackage{hyperref}
    \usepackage{longtable} % longtable support required by pandoc >1.10
    \usepackage{booktabs}  % table support for pandoc > 1.12.2
    \usepackage[inline]{enumitem} % IRkernel/repr support (it uses the enumerate* environment)
    \usepackage[normalem]{ulem} % ulem is needed to support strikethroughs (\sout)
                                % normalem makes italics be italics, not underlines
    

    
    
    % Colors for the hyperref package
    \definecolor{urlcolor}{rgb}{0,.145,.698}
    \definecolor{linkcolor}{rgb}{.71,0.21,0.01}
    \definecolor{citecolor}{rgb}{.12,.54,.11}

    % ANSI colors
    \definecolor{ansi-black}{HTML}{3E424D}
    \definecolor{ansi-black-intense}{HTML}{282C36}
    \definecolor{ansi-red}{HTML}{E75C58}
    \definecolor{ansi-red-intense}{HTML}{B22B31}
    \definecolor{ansi-green}{HTML}{00A250}
    \definecolor{ansi-green-intense}{HTML}{007427}
    \definecolor{ansi-yellow}{HTML}{DDB62B}
    \definecolor{ansi-yellow-intense}{HTML}{B27D12}
    \definecolor{ansi-blue}{HTML}{208FFB}
    \definecolor{ansi-blue-intense}{HTML}{0065CA}
    \definecolor{ansi-magenta}{HTML}{D160C4}
    \definecolor{ansi-magenta-intense}{HTML}{A03196}
    \definecolor{ansi-cyan}{HTML}{60C6C8}
    \definecolor{ansi-cyan-intense}{HTML}{258F8F}
    \definecolor{ansi-white}{HTML}{C5C1B4}
    \definecolor{ansi-white-intense}{HTML}{A1A6B2}

    % commands and environments needed by pandoc snippets
    % extracted from the output of `pandoc -s`
    \providecommand{\tightlist}{%
      \setlength{\itemsep}{0pt}\setlength{\parskip}{0pt}}
    \DefineVerbatimEnvironment{Highlighting}{Verbatim}{commandchars=\\\{\}}
    % Add ',fontsize=\small' for more characters per line
    \newenvironment{Shaded}{}{}
    \newcommand{\KeywordTok}[1]{\textcolor[rgb]{0.00,0.44,0.13}{\textbf{{#1}}}}
    \newcommand{\DataTypeTok}[1]{\textcolor[rgb]{0.56,0.13,0.00}{{#1}}}
    \newcommand{\DecValTok}[1]{\textcolor[rgb]{0.25,0.63,0.44}{{#1}}}
    \newcommand{\BaseNTok}[1]{\textcolor[rgb]{0.25,0.63,0.44}{{#1}}}
    \newcommand{\FloatTok}[1]{\textcolor[rgb]{0.25,0.63,0.44}{{#1}}}
    \newcommand{\CharTok}[1]{\textcolor[rgb]{0.25,0.44,0.63}{{#1}}}
    \newcommand{\StringTok}[1]{\textcolor[rgb]{0.25,0.44,0.63}{{#1}}}
    \newcommand{\CommentTok}[1]{\textcolor[rgb]{0.38,0.63,0.69}{\textit{{#1}}}}
    \newcommand{\OtherTok}[1]{\textcolor[rgb]{0.00,0.44,0.13}{{#1}}}
    \newcommand{\AlertTok}[1]{\textcolor[rgb]{1.00,0.00,0.00}{\textbf{{#1}}}}
    \newcommand{\FunctionTok}[1]{\textcolor[rgb]{0.02,0.16,0.49}{{#1}}}
    \newcommand{\RegionMarkerTok}[1]{{#1}}
    \newcommand{\ErrorTok}[1]{\textcolor[rgb]{1.00,0.00,0.00}{\textbf{{#1}}}}
    \newcommand{\NormalTok}[1]{{#1}}
    
    % Additional commands for more recent versions of Pandoc
    \newcommand{\ConstantTok}[1]{\textcolor[rgb]{0.53,0.00,0.00}{{#1}}}
    \newcommand{\SpecialCharTok}[1]{\textcolor[rgb]{0.25,0.44,0.63}{{#1}}}
    \newcommand{\VerbatimStringTok}[1]{\textcolor[rgb]{0.25,0.44,0.63}{{#1}}}
    \newcommand{\SpecialStringTok}[1]{\textcolor[rgb]{0.73,0.40,0.53}{{#1}}}
    \newcommand{\ImportTok}[1]{{#1}}
    \newcommand{\DocumentationTok}[1]{\textcolor[rgb]{0.73,0.13,0.13}{\textit{{#1}}}}
    \newcommand{\AnnotationTok}[1]{\textcolor[rgb]{0.38,0.63,0.69}{\textbf{\textit{{#1}}}}}
    \newcommand{\CommentVarTok}[1]{\textcolor[rgb]{0.38,0.63,0.69}{\textbf{\textit{{#1}}}}}
    \newcommand{\VariableTok}[1]{\textcolor[rgb]{0.10,0.09,0.49}{{#1}}}
    \newcommand{\ControlFlowTok}[1]{\textcolor[rgb]{0.00,0.44,0.13}{\textbf{{#1}}}}
    \newcommand{\OperatorTok}[1]{\textcolor[rgb]{0.40,0.40,0.40}{{#1}}}
    \newcommand{\BuiltInTok}[1]{{#1}}
    \newcommand{\ExtensionTok}[1]{{#1}}
    \newcommand{\PreprocessorTok}[1]{\textcolor[rgb]{0.74,0.48,0.00}{{#1}}}
    \newcommand{\AttributeTok}[1]{\textcolor[rgb]{0.49,0.56,0.16}{{#1}}}
    \newcommand{\InformationTok}[1]{\textcolor[rgb]{0.38,0.63,0.69}{\textbf{\textit{{#1}}}}}
    \newcommand{\WarningTok}[1]{\textcolor[rgb]{0.38,0.63,0.69}{\textbf{\textit{{#1}}}}}
    
    
    % Define a nice break command that doesn't care if a line doesn't already
    % exist.
    \def\br{\hspace*{\fill} \\* }
    % Math Jax compatability definitions
    \def\gt{>}
    \def\lt{<}
    % Document parameters
    \title{homework-02}
    
    
    

    % Pygments definitions
    
\makeatletter
\def\PY@reset{\let\PY@it=\relax \let\PY@bf=\relax%
    \let\PY@ul=\relax \let\PY@tc=\relax%
    \let\PY@bc=\relax \let\PY@ff=\relax}
\def\PY@tok#1{\csname PY@tok@#1\endcsname}
\def\PY@toks#1+{\ifx\relax#1\empty\else%
    \PY@tok{#1}\expandafter\PY@toks\fi}
\def\PY@do#1{\PY@bc{\PY@tc{\PY@ul{%
    \PY@it{\PY@bf{\PY@ff{#1}}}}}}}
\def\PY#1#2{\PY@reset\PY@toks#1+\relax+\PY@do{#2}}

\expandafter\def\csname PY@tok@w\endcsname{\def\PY@tc##1{\textcolor[rgb]{0.73,0.73,0.73}{##1}}}
\expandafter\def\csname PY@tok@c\endcsname{\let\PY@it=\textit\def\PY@tc##1{\textcolor[rgb]{0.25,0.50,0.50}{##1}}}
\expandafter\def\csname PY@tok@cp\endcsname{\def\PY@tc##1{\textcolor[rgb]{0.74,0.48,0.00}{##1}}}
\expandafter\def\csname PY@tok@k\endcsname{\let\PY@bf=\textbf\def\PY@tc##1{\textcolor[rgb]{0.00,0.50,0.00}{##1}}}
\expandafter\def\csname PY@tok@kp\endcsname{\def\PY@tc##1{\textcolor[rgb]{0.00,0.50,0.00}{##1}}}
\expandafter\def\csname PY@tok@kt\endcsname{\def\PY@tc##1{\textcolor[rgb]{0.69,0.00,0.25}{##1}}}
\expandafter\def\csname PY@tok@o\endcsname{\def\PY@tc##1{\textcolor[rgb]{0.40,0.40,0.40}{##1}}}
\expandafter\def\csname PY@tok@ow\endcsname{\let\PY@bf=\textbf\def\PY@tc##1{\textcolor[rgb]{0.67,0.13,1.00}{##1}}}
\expandafter\def\csname PY@tok@nb\endcsname{\def\PY@tc##1{\textcolor[rgb]{0.00,0.50,0.00}{##1}}}
\expandafter\def\csname PY@tok@nf\endcsname{\def\PY@tc##1{\textcolor[rgb]{0.00,0.00,1.00}{##1}}}
\expandafter\def\csname PY@tok@nc\endcsname{\let\PY@bf=\textbf\def\PY@tc##1{\textcolor[rgb]{0.00,0.00,1.00}{##1}}}
\expandafter\def\csname PY@tok@nn\endcsname{\let\PY@bf=\textbf\def\PY@tc##1{\textcolor[rgb]{0.00,0.00,1.00}{##1}}}
\expandafter\def\csname PY@tok@ne\endcsname{\let\PY@bf=\textbf\def\PY@tc##1{\textcolor[rgb]{0.82,0.25,0.23}{##1}}}
\expandafter\def\csname PY@tok@nv\endcsname{\def\PY@tc##1{\textcolor[rgb]{0.10,0.09,0.49}{##1}}}
\expandafter\def\csname PY@tok@no\endcsname{\def\PY@tc##1{\textcolor[rgb]{0.53,0.00,0.00}{##1}}}
\expandafter\def\csname PY@tok@nl\endcsname{\def\PY@tc##1{\textcolor[rgb]{0.63,0.63,0.00}{##1}}}
\expandafter\def\csname PY@tok@ni\endcsname{\let\PY@bf=\textbf\def\PY@tc##1{\textcolor[rgb]{0.60,0.60,0.60}{##1}}}
\expandafter\def\csname PY@tok@na\endcsname{\def\PY@tc##1{\textcolor[rgb]{0.49,0.56,0.16}{##1}}}
\expandafter\def\csname PY@tok@nt\endcsname{\let\PY@bf=\textbf\def\PY@tc##1{\textcolor[rgb]{0.00,0.50,0.00}{##1}}}
\expandafter\def\csname PY@tok@nd\endcsname{\def\PY@tc##1{\textcolor[rgb]{0.67,0.13,1.00}{##1}}}
\expandafter\def\csname PY@tok@s\endcsname{\def\PY@tc##1{\textcolor[rgb]{0.73,0.13,0.13}{##1}}}
\expandafter\def\csname PY@tok@sd\endcsname{\let\PY@it=\textit\def\PY@tc##1{\textcolor[rgb]{0.73,0.13,0.13}{##1}}}
\expandafter\def\csname PY@tok@si\endcsname{\let\PY@bf=\textbf\def\PY@tc##1{\textcolor[rgb]{0.73,0.40,0.53}{##1}}}
\expandafter\def\csname PY@tok@se\endcsname{\let\PY@bf=\textbf\def\PY@tc##1{\textcolor[rgb]{0.73,0.40,0.13}{##1}}}
\expandafter\def\csname PY@tok@sr\endcsname{\def\PY@tc##1{\textcolor[rgb]{0.73,0.40,0.53}{##1}}}
\expandafter\def\csname PY@tok@ss\endcsname{\def\PY@tc##1{\textcolor[rgb]{0.10,0.09,0.49}{##1}}}
\expandafter\def\csname PY@tok@sx\endcsname{\def\PY@tc##1{\textcolor[rgb]{0.00,0.50,0.00}{##1}}}
\expandafter\def\csname PY@tok@m\endcsname{\def\PY@tc##1{\textcolor[rgb]{0.40,0.40,0.40}{##1}}}
\expandafter\def\csname PY@tok@gh\endcsname{\let\PY@bf=\textbf\def\PY@tc##1{\textcolor[rgb]{0.00,0.00,0.50}{##1}}}
\expandafter\def\csname PY@tok@gu\endcsname{\let\PY@bf=\textbf\def\PY@tc##1{\textcolor[rgb]{0.50,0.00,0.50}{##1}}}
\expandafter\def\csname PY@tok@gd\endcsname{\def\PY@tc##1{\textcolor[rgb]{0.63,0.00,0.00}{##1}}}
\expandafter\def\csname PY@tok@gi\endcsname{\def\PY@tc##1{\textcolor[rgb]{0.00,0.63,0.00}{##1}}}
\expandafter\def\csname PY@tok@gr\endcsname{\def\PY@tc##1{\textcolor[rgb]{1.00,0.00,0.00}{##1}}}
\expandafter\def\csname PY@tok@ge\endcsname{\let\PY@it=\textit}
\expandafter\def\csname PY@tok@gs\endcsname{\let\PY@bf=\textbf}
\expandafter\def\csname PY@tok@gp\endcsname{\let\PY@bf=\textbf\def\PY@tc##1{\textcolor[rgb]{0.00,0.00,0.50}{##1}}}
\expandafter\def\csname PY@tok@go\endcsname{\def\PY@tc##1{\textcolor[rgb]{0.53,0.53,0.53}{##1}}}
\expandafter\def\csname PY@tok@gt\endcsname{\def\PY@tc##1{\textcolor[rgb]{0.00,0.27,0.87}{##1}}}
\expandafter\def\csname PY@tok@err\endcsname{\def\PY@bc##1{\setlength{\fboxsep}{0pt}\fcolorbox[rgb]{1.00,0.00,0.00}{1,1,1}{\strut ##1}}}
\expandafter\def\csname PY@tok@kc\endcsname{\let\PY@bf=\textbf\def\PY@tc##1{\textcolor[rgb]{0.00,0.50,0.00}{##1}}}
\expandafter\def\csname PY@tok@kd\endcsname{\let\PY@bf=\textbf\def\PY@tc##1{\textcolor[rgb]{0.00,0.50,0.00}{##1}}}
\expandafter\def\csname PY@tok@kn\endcsname{\let\PY@bf=\textbf\def\PY@tc##1{\textcolor[rgb]{0.00,0.50,0.00}{##1}}}
\expandafter\def\csname PY@tok@kr\endcsname{\let\PY@bf=\textbf\def\PY@tc##1{\textcolor[rgb]{0.00,0.50,0.00}{##1}}}
\expandafter\def\csname PY@tok@bp\endcsname{\def\PY@tc##1{\textcolor[rgb]{0.00,0.50,0.00}{##1}}}
\expandafter\def\csname PY@tok@fm\endcsname{\def\PY@tc##1{\textcolor[rgb]{0.00,0.00,1.00}{##1}}}
\expandafter\def\csname PY@tok@vc\endcsname{\def\PY@tc##1{\textcolor[rgb]{0.10,0.09,0.49}{##1}}}
\expandafter\def\csname PY@tok@vg\endcsname{\def\PY@tc##1{\textcolor[rgb]{0.10,0.09,0.49}{##1}}}
\expandafter\def\csname PY@tok@vi\endcsname{\def\PY@tc##1{\textcolor[rgb]{0.10,0.09,0.49}{##1}}}
\expandafter\def\csname PY@tok@vm\endcsname{\def\PY@tc##1{\textcolor[rgb]{0.10,0.09,0.49}{##1}}}
\expandafter\def\csname PY@tok@sa\endcsname{\def\PY@tc##1{\textcolor[rgb]{0.73,0.13,0.13}{##1}}}
\expandafter\def\csname PY@tok@sb\endcsname{\def\PY@tc##1{\textcolor[rgb]{0.73,0.13,0.13}{##1}}}
\expandafter\def\csname PY@tok@sc\endcsname{\def\PY@tc##1{\textcolor[rgb]{0.73,0.13,0.13}{##1}}}
\expandafter\def\csname PY@tok@dl\endcsname{\def\PY@tc##1{\textcolor[rgb]{0.73,0.13,0.13}{##1}}}
\expandafter\def\csname PY@tok@s2\endcsname{\def\PY@tc##1{\textcolor[rgb]{0.73,0.13,0.13}{##1}}}
\expandafter\def\csname PY@tok@sh\endcsname{\def\PY@tc##1{\textcolor[rgb]{0.73,0.13,0.13}{##1}}}
\expandafter\def\csname PY@tok@s1\endcsname{\def\PY@tc##1{\textcolor[rgb]{0.73,0.13,0.13}{##1}}}
\expandafter\def\csname PY@tok@mb\endcsname{\def\PY@tc##1{\textcolor[rgb]{0.40,0.40,0.40}{##1}}}
\expandafter\def\csname PY@tok@mf\endcsname{\def\PY@tc##1{\textcolor[rgb]{0.40,0.40,0.40}{##1}}}
\expandafter\def\csname PY@tok@mh\endcsname{\def\PY@tc##1{\textcolor[rgb]{0.40,0.40,0.40}{##1}}}
\expandafter\def\csname PY@tok@mi\endcsname{\def\PY@tc##1{\textcolor[rgb]{0.40,0.40,0.40}{##1}}}
\expandafter\def\csname PY@tok@il\endcsname{\def\PY@tc##1{\textcolor[rgb]{0.40,0.40,0.40}{##1}}}
\expandafter\def\csname PY@tok@mo\endcsname{\def\PY@tc##1{\textcolor[rgb]{0.40,0.40,0.40}{##1}}}
\expandafter\def\csname PY@tok@ch\endcsname{\let\PY@it=\textit\def\PY@tc##1{\textcolor[rgb]{0.25,0.50,0.50}{##1}}}
\expandafter\def\csname PY@tok@cm\endcsname{\let\PY@it=\textit\def\PY@tc##1{\textcolor[rgb]{0.25,0.50,0.50}{##1}}}
\expandafter\def\csname PY@tok@cpf\endcsname{\let\PY@it=\textit\def\PY@tc##1{\textcolor[rgb]{0.25,0.50,0.50}{##1}}}
\expandafter\def\csname PY@tok@c1\endcsname{\let\PY@it=\textit\def\PY@tc##1{\textcolor[rgb]{0.25,0.50,0.50}{##1}}}
\expandafter\def\csname PY@tok@cs\endcsname{\let\PY@it=\textit\def\PY@tc##1{\textcolor[rgb]{0.25,0.50,0.50}{##1}}}

\def\PYZbs{\char`\\}
\def\PYZus{\char`\_}
\def\PYZob{\char`\{}
\def\PYZcb{\char`\}}
\def\PYZca{\char`\^}
\def\PYZam{\char`\&}
\def\PYZlt{\char`\<}
\def\PYZgt{\char`\>}
\def\PYZsh{\char`\#}
\def\PYZpc{\char`\%}
\def\PYZdl{\char`\$}
\def\PYZhy{\char`\-}
\def\PYZsq{\char`\'}
\def\PYZdq{\char`\"}
\def\PYZti{\char`\~}
% for compatibility with earlier versions
\def\PYZat{@}
\def\PYZlb{[}
\def\PYZrb{]}
\makeatother


    % Exact colors from NB
    \definecolor{incolor}{rgb}{0.0, 0.0, 0.5}
    \definecolor{outcolor}{rgb}{0.545, 0.0, 0.0}



    
    % Prevent overflowing lines due to hard-to-break entities
    \sloppy 
    % Setup hyperref package
    \hypersetup{
      breaklinks=true,  % so long urls are correctly broken across lines
      colorlinks=true,
      urlcolor=urlcolor,
      linkcolor=linkcolor,
      citecolor=citecolor,
      }
    % Slightly bigger margins than the latex defaults
    
    \geometry{verbose,tmargin=1in,bmargin=1in,lmargin=1in,rmargin=1in}
    
    

    \begin{document}
    
    
    \maketitle
    
    

    
    \hypertarget{homwork-2}{%
\section{Homwork 2}\label{homwork-2}}

\textbf{Name:} Tommy Bojanin - Bojanits

    \begin{Verbatim}[commandchars=\\\{\}]
{\color{incolor}In [{\color{incolor}102}]:} \PY{k+kn}{from} \PY{n+nn}{\PYZus{}\PYZus{}future\PYZus{}\PYZus{}} \PY{k}{import} \PY{n}{print\PYZus{}function}\PY{p}{,} \PY{n}{division}
\end{Verbatim}


    \hypertarget{using-this-notebook}{%
\section{Using this notebook}\label{using-this-notebook}}

As a reminder, this is a Jupyter notebook, you cannot edit the version
you see online in the \texttt{git} repo. You must \emph{use git} to
checkout the repo, and then use \texttt{jupyter\ notebook} to edit your
local copy of this. You are encouraged to \emph{fork} my repo so you
save and track your own changes.

    \hypertarget{instructions}{%
\subsection{Instructions}\label{instructions}}

\begin{itemize}
\tightlist
\item
  You \emph{may} submit very neatly organized
  \emph{\textbf{handwritten}} math, or you may do the math in the cells
  of this notebook using \(\LaTeX\). With \(\LaTeX\), you may follow the
  example in this cell, or you can put \texttt{\%\%latex} at the top of
  any cell.\\
\item
  In the last problem set most submissions were clear and
  reasonablypresented, but I have my preferences for what I like to see
  from students. In particular, I am a big fan of the 3-column
  \texttt{align} environment, with the last column used as an
  explanation. In particular, this: \[ E(\mathbf{w}) 
   = \sum_{n=1}^N \{y(x_n |\mathbf{w}) - t_n\}^2
   = \sum_{n=1}^N \left\{\sum_{j=0}^M x_n^jw_j - t_n\right\}^2 \\
   \frac{\partial}{\partial w_i} E(\mathbf{w}) 
   = \sum_{n=1}^N \left\{\sum_{j=0}^M x_n^jw_j - t_n\right\}x_n^i 
   = \sum_{n=1}^N \sum_{j=0}^M x_n^jx_n^iw_j - \sum_{n=1}^N t_n x^i 
   = \sum_{j=0}^M \underbrace{\left\{\sum_{n=1}^N x_n^{j+i}\right\}}_{A_{ij}}w_j - \underbrace{\sum_{n=1}^N t_n x^i}_{T_i} \]
\end{itemize}

is quite hard for me to follow, even if it is correct. A \textbf{better}
answer (for homework, not for formal papers) uses the \texttt{align}
environment:

\begin{quote}
{[}Explaining the idea, that we want the gradient to be zero, etc{]} \[ 
\begin{align}
E(\mathbf{w}) 
&= \sum_{n=1}^N \{y(x_n |\mathbf{w}) - t_n\}^2  & \text{by dfn. (1.1) of PRML}  \\
&= \sum_{n=1}^N \left\{\sum_{j=0}^M x_n^jw_j - t_n\right\}^2 & \text{subst. (1.2) into (1.1)} \hspace{1em}\\
\end{align} 
\] {[}Explain that we want
\(\frac{\partial}{\partial w_i} E = 0, \forall i\){]} \[ 
\begin{align}
\frac{\partial}{\partial w_i} E(\mathbf{w})  
&= \sum_{n=1}^N \left\{\sum_{j=0}^M x_n^jw_j - t_n\right\}x_n^i   & \text{applying the chain rule [details..]}  \\ 
&= \sum_{n=1}^N \sum_{j=0}^M x_n^jx_n^iw_j - \sum_{n=1}^N t_n x^i & \text{distributing $x_n^i$, and $\sum$.. 2 steps would be better} \\
&= \sum_{j=0}^M \underbrace{\left\{\sum_{n=1}^N x_n^{j+i}\right\}}_{A_{ij}}w_j - \underbrace{\sum_{n=1}^N t_nx^i}_{T_i}  & \text{add terms in a different order, group $x^ix^j=x^{i+j}$} \\
&=0 \implies \sum_{j=0}^M A_{ij}w_j = T_i \\
\end{align} 
\] Q.E.D
\end{quote}

The key difference is that the latter one is: * A step-by-step process
that proceeds from what is given, \(E(\mathbf{w})\), to what we aim to
show, \(\sum_{j=0}^M A_{ij}w_j = T_i \implies \mathbf{w}=\mathbf{w}^*\).
* The rationale behind each step is explained, in one of two ways; * as
a comment in the right margin letting me know why the step is valid * or
as plain-old-english (POE) explaining the intent of a set of steps,
\emph{before} those steps are carried out.

\begin{quote}
\textbf{NOTE:} You may edit this cell to peak at the \(\LaTeX\) I used
to type out my answers. In its final form the \(\LaTeX\) can seem
intimidating, but most of it is copy-paste-modify for each step.
\end{quote}

    \hypertarget{problems}{%
\subsection{Problems}\label{problems}}

    1. \emph{{[}4 pts{]}} Do question 2.57 of PRML. * For practice, try to
solve 2.56 first (do not submit your answers to that one). * \_
\textbf{2.57(*)} Verify that the multivariate gaussian distributions ca
be cast in exponential family form (2.194) and derive expressions for
\(\boldsymbol{\eta}\), \(\mathbf{u}(\mathbf{x})\), \(h(\mathbf{x})\),
and \(g(\boldsymbol{\eta})\) analogous to (2.220)-(2.223)\_

    \(\textbf{Did on paper, attached.}\)

    2. \emph{{[}4 pts{]}} Reproduce figure 2.8. * Your Gaussian's do not
have to be identical, but the covariance matrices should be (a) general
{[}not diagonal or isometric{]}, (b) diagonal, and (c) proportional to
the identity matrix.

    \begin{Verbatim}[commandchars=\\\{\}]
{\color{incolor}In [{\color{incolor}96}]:} \PY{c+c1}{\PYZsh{}Used your example from the lab, spent most of my time on problem 3 since it had data to use}
         \PY{k+kn}{import} \PY{n+nn}{matplotlib}\PY{n+nn}{.}\PY{n+nn}{pyplot} \PY{k}{as} \PY{n+nn}{plt}
         \PY{k+kn}{import} \PY{n+nn}{matplotlib} \PY{k}{as} \PY{n+nn}{mpl}
         \PY{k+kn}{import} \PY{n+nn}{numpy} \PY{k}{as} \PY{n+nn}{np}
         \PY{k+kn}{from} \PY{n+nn}{scipy} \PY{k}{import} \PY{n}{linalg}
         \PY{k+kn}{import} \PY{n+nn}{scipy}
         \PY{k+kn}{import} \PY{n+nn}{scipy}\PY{n+nn}{.}\PY{n+nn}{stats} \PY{k}{as} \PY{n+nn}{st}
         
         
         
         \PY{n}{x\PYZus{}1}\PY{p}{,} \PY{n}{x\PYZus{}2} \PY{o}{=} \PY{n}{mgrid}\PY{p}{[}\PY{o}{\PYZhy{}}\PY{l+m+mi}{3}\PY{p}{:}\PY{l+m+mi}{3}\PY{p}{:}\PY{l+m+mi}{100}\PY{n}{j}\PY{p}{,} \PY{o}{\PYZhy{}}\PY{l+m+mi}{3}\PY{p}{:}\PY{l+m+mi}{3}\PY{p}{:}\PY{l+m+mi}{100}\PY{n}{j}\PY{p}{]}
         \PY{n}{y} \PY{o}{=} \PY{n}{zeros\PYZus{}like}\PY{p}{(}\PY{n}{x\PYZus{}1}\PY{p}{)}
         
         \PY{n}{y}\PY{o}{.}\PY{n}{flat} \PY{o}{=} \PY{p}{[}\PY{n}{scipy}\PY{o}{.}\PY{n}{stats}\PY{o}{.}\PY{n}{multivariate\PYZus{}normal}\PY{o}{.}\PY{n}{pdf}\PY{p}{(}\PY{n}{x}\PY{p}{,}\PY{n}{cov}\PY{o}{=}\PY{n}{eye}\PY{p}{(}\PY{l+m+mi}{2}\PY{p}{)}\PY{p}{)} \PY{k}{for} \PY{n}{x} \PY{o+ow}{in} \PY{n+nb}{zip}\PY{p}{(}\PY{n}{x\PYZus{}1}\PY{o}{.}\PY{n}{flat}\PY{p}{,} \PY{n}{x\PYZus{}2}\PY{o}{.}\PY{n}{flat}\PY{p}{)}\PY{p}{]}
         
         \PY{n}{imshow}\PY{p}{(}\PY{n}{y}\PY{p}{,} \PY{n}{extent}\PY{o}{=}\PY{p}{(}\PY{o}{\PYZhy{}}\PY{l+m+mi}{5}\PY{p}{,}\PY{l+m+mi}{5}\PY{p}{,}\PY{o}{\PYZhy{}}\PY{l+m+mi}{5}\PY{p}{,}\PY{l+m+mi}{5}\PY{p}{)}\PY{p}{,} \PY{n}{origin}\PY{o}{=}\PY{l+s+s1}{\PYZsq{}}\PY{l+s+s1}{lower}\PY{l+s+s1}{\PYZsq{}}\PY{p}{,} \PY{n}{alpha}\PY{o}{=}\PY{l+m+mf}{0.0}\PY{p}{)}
         \PY{n}{clabel}\PY{p}{(}\PY{n}{contour}\PY{p}{(}\PY{n}{x\PYZus{}1}\PY{p}{,} \PY{n}{x\PYZus{}2}\PY{p}{,} \PY{n}{y}\PY{p}{,} \PY{n}{levels}\PY{o}{=} \PY{p}{[}\PY{o}{.}\PY{l+m+mi}{5}\PY{p}{,} \PY{l+m+mf}{0.002}\PY{p}{,} \PY{l+m+mf}{0.022}\PY{p}{,} \PY{l+m+mf}{0.097}\PY{p}{]}\PY{p}{)}\PY{p}{)}
\end{Verbatim}


\begin{Verbatim}[commandchars=\\\{\}]
{\color{outcolor}Out[{\color{outcolor}96}]:} <a list of 3 text.Text objects>
\end{Verbatim}
            
    \begin{center}
    \adjustimage{max size={0.9\linewidth}{0.9\paperheight}}{output_8_1.png}
    \end{center}
    { \hspace*{\fill} \\}
    
    3. \emph{{[}4 points{]}} Reproduce figure 2.21. * The \emph{old
faithful} data is describe in appendix A, you should be able to download
it and open it. * You can (and, for now, you should) use a function from
\texttt{sklearn} in order to fit a mixture of Gaussians to your data. I
beleive you can find what you need
\href{http://scikit-learn.org/stable/modules/mixture.html}{here}

    \textbf{NOTE:} In order to help you out, I will get you started by
downloading the data and plotting it

    \begin{Verbatim}[commandchars=\\\{\}]
{\color{incolor}In [{\color{incolor}97}]:} \PY{c+c1}{\PYZsh{} Download the data}
         \PY{k+kn}{import} \PY{n+nn}{urllib}\PY{n+nn}{.}\PY{n+nn}{request}
         \PY{k+kn}{import} \PY{n+nn}{itertools}
         \PY{k+kn}{from} \PY{n+nn}{sklearn} \PY{k}{import} \PY{n}{mixture}
         \PY{n}{np}\PY{o}{.}\PY{n}{random}\PY{o}{.}\PY{n}{seed}\PY{p}{(}\PY{l+m+mi}{127}\PY{p}{)}
         
         \PY{n}{path} \PY{o}{=} \PY{l+s+s1}{\PYZsq{}}\PY{l+s+s1}{faithful.txt}\PY{l+s+s1}{\PYZsq{}}
         \PY{n}{remote} \PY{o}{=} \PY{n}{urllib}\PY{o}{.}\PY{n}{request}\PY{o}{.}\PY{n}{urlopen}\PY{p}{(}\PY{l+s+s1}{\PYZsq{}}\PY{l+s+s1}{https://raw.githubusercontent.com/aidiary/PRML/master/ch9/faithful.txt}\PY{l+s+s1}{\PYZsq{}}\PY{p}{)}
         \PY{k}{with} \PY{n+nb}{open}\PY{p}{(}\PY{l+s+s1}{\PYZsq{}}\PY{l+s+s1}{faithful.txt}\PY{l+s+s1}{\PYZsq{}}\PY{p}{,} \PY{l+s+s1}{\PYZsq{}}\PY{l+s+s1}{wb}\PY{l+s+s1}{\PYZsq{}}\PY{p}{)} \PY{k}{as} \PY{n}{f}\PY{p}{:}
             \PY{n}{f}\PY{o}{.}\PY{n}{write}\PY{p}{(}\PY{n}{remote}\PY{o}{.}\PY{n}{read}\PY{p}{(}\PY{p}{)}\PY{p}{)}
\end{Verbatim}


    \begin{Verbatim}[commandchars=\\\{\}]
{\color{incolor}In [{\color{incolor}98}]:} \PY{c+c1}{\PYZsh{} Read it in and inspect it}
         \PY{k+kn}{import} \PY{n+nn}{pandas}
         \PY{k+kn}{from} \PY{n+nn}{IPython}\PY{n+nn}{.}\PY{n+nn}{display} \PY{k}{import} \PY{n}{display}
         \PY{n}{old\PYZus{}faithful} \PY{o}{=} \PY{n}{pandas}\PY{o}{.}\PY{n}{read\PYZus{}csv}\PY{p}{(}\PY{n}{path}\PY{p}{,} \PY{n}{sep}\PY{o}{=}\PY{l+s+s1}{\PYZsq{}}\PY{l+s+s1}{ }\PY{l+s+s1}{\PYZsq{}}\PY{p}{,} \PY{n}{names}\PY{o}{=}\PY{p}{[}\PY{l+s+s1}{\PYZsq{}}\PY{l+s+s1}{duration}\PY{l+s+s1}{\PYZsq{}}\PY{p}{,} \PY{l+s+s1}{\PYZsq{}}\PY{l+s+s1}{next}\PY{l+s+s1}{\PYZsq{}}\PY{p}{]}\PY{p}{)}
         \PY{n}{display}\PY{p}{(}\PY{n}{old\PYZus{}faithful}\PY{p}{)}
\end{Verbatim}


    
    \begin{verbatim}
     duration  next
0       3.600  79.0
1       1.800  54.0
2       3.333  74.0
3       2.283  62.0
4       4.533  85.0
5       2.883  55.0
6       4.700  88.0
7       3.600  85.0
8       1.950  51.0
9       4.350  85.0
10      1.833  54.0
11      3.917  84.0
12      4.200  78.0
13      1.750  47.0
14      4.700  83.0
15      2.167  52.0
16      1.750  62.0
17      4.800  84.0
18      1.600  52.0
19      4.250  79.0
20      1.800  51.0
21      1.750  47.0
22      3.450  78.0
23      3.067  69.0
24      4.533  74.0
25      3.600  83.0
26      1.967  55.0
27      4.083  76.0
28      3.850  78.0
29      4.433  79.0
..        ...   ...
242     4.933  86.0
243     2.900  63.0
244     4.583  85.0
245     3.833  82.0
246     2.083  57.0
247     4.367  82.0
248     2.133  67.0
249     4.350  74.0
250     2.200  54.0
251     4.450  83.0
252     3.567  73.0
253     4.500  73.0
254     4.150  88.0
255     3.817  80.0
256     3.917  71.0
257     4.450  83.0
258     2.000  56.0
259     4.283  79.0
260     4.767  78.0
261     4.533  84.0
262     1.850  58.0
263     4.250  83.0
264     1.983  43.0
265     2.250  60.0
266     4.750  75.0
267     4.117  81.0
268     2.150  46.0
269     4.417  90.0
270     1.817  46.0
271     4.467  74.0

[272 rows x 2 columns]
    \end{verbatim}

    
    \begin{Verbatim}[commandchars=\\\{\}]
{\color{incolor}In [{\color{incolor}99}]:} \PY{c+c1}{\PYZsh{}used the sklearn \PYZdq{}here\PYZdq{} link}
         \PY{k}{def} \PY{n+nf}{plot\PYZus{}mix}\PY{p}{(}\PY{n}{mix}\PY{p}{)}\PY{p}{:}
             \PY{n}{mask} \PY{o}{=} \PY{n}{mix}\PY{o}{.}\PY{n}{predict}\PY{p}{(}\PY{n}{data}\PY{p}{)}
             \PY{n}{xx} \PY{o}{=} \PY{n}{np}\PY{o}{.}\PY{n}{linspace}\PY{p}{(}\PY{l+m+mi}{0}\PY{p}{,} \PY{l+m+mi}{6}\PY{p}{)} 
             \PY{n}{yy} \PY{o}{=} \PY{n}{np}\PY{o}{.}\PY{n}{linspace}\PY{p}{(}\PY{l+m+mi}{40}\PY{p}{,} \PY{l+m+mi}{100}\PY{p}{)} 
             \PY{n}{X}\PY{p}{,} \PY{n}{Y} \PY{o}{=} \PY{n}{np}\PY{o}{.}\PY{n}{meshgrid}\PY{p}{(}\PY{n}{xx}\PY{p}{,} \PY{n}{yy}\PY{p}{)}
             \PY{n}{XX} \PY{o}{=} \PY{n}{np}\PY{o}{.}\PY{n}{c\PYZus{}}\PY{p}{[}\PY{n}{X}\PY{o}{.}\PY{n}{ravel}\PY{p}{(}\PY{p}{)}\PY{p}{,} \PY{n}{Y}\PY{o}{.}\PY{n}{ravel}\PY{p}{(}\PY{p}{)}\PY{p}{]}
             \PY{n}{Z} \PY{o}{=}  \PY{n}{mix}\PY{o}{.}\PY{n}{score\PYZus{}samples}\PY{p}{(}\PY{n}{XX}\PY{p}{)}
             \PY{n}{Z} \PY{o}{=} \PY{n}{Z}\PY{o}{.}\PY{n}{reshape}\PY{p}{(}\PY{n}{X}\PY{o}{.}\PY{n}{shape}\PY{p}{)}
         
             \PY{n}{CS} \PY{o}{=} \PY{n}{plt}\PY{o}{.}\PY{n}{contour}\PY{p}{(}\PY{n}{X}\PY{p}{,} \PY{n}{Y}\PY{p}{,} \PY{n}{Z}\PY{p}{)}
             \PY{n}{CB} \PY{o}{=} \PY{n}{plt}\PY{o}{.}\PY{n}{colorbar}\PY{p}{(}\PY{n}{CS}\PY{p}{,} \PY{n}{shrink}\PY{o}{=}\PY{l+m+mf}{0.8}\PY{p}{,} \PY{n}{extend}\PY{o}{=}\PY{l+s+s1}{\PYZsq{}}\PY{l+s+s1}{both}\PY{l+s+s1}{\PYZsq{}}\PY{p}{)}
             \PY{n}{plt}\PY{o}{.}\PY{n}{scatter}\PY{p}{(}\PY{n}{old\PYZus{}faithful}\PY{o}{.}\PY{n}{duration}\PY{p}{,} \PY{n}{old\PYZus{}faithful}\PY{o}{.}\PY{n}{next}\PY{p}{,} \PY{n}{c}\PY{o}{=}\PY{n}{mask}\PY{p}{)}\PY{p}{;}
         
         \PY{c+c1}{\PYZsh{} Plot the data \PYZhy{}\PYZhy{}\PYZgt{} note that I convert from a DataFrame to an array }
         \PY{n}{data} \PY{o}{=} \PY{n}{np}\PY{o}{.}\PY{n}{array}\PY{p}{(}\PY{n}{old\PYZus{}faithful}\PY{p}{)}
         \PY{n}{scatter}\PY{p}{(}\PY{n}{data}\PY{p}{[}\PY{p}{:}\PY{p}{,}\PY{l+m+mi}{0}\PY{p}{]}\PY{p}{,} \PY{n}{data}\PY{p}{[}\PY{p}{:}\PY{p}{,}\PY{l+m+mi}{1}\PY{p}{]}\PY{p}{,} \PY{n}{color}\PY{o}{=}\PY{l+s+s1}{\PYZsq{}}\PY{l+s+s1}{lightgreen}\PY{l+s+s1}{\PYZsq{}}\PY{p}{)}
         \PY{n}{gaussian\PYZus{}fit} \PY{o}{=} \PY{n}{mixture}\PY{o}{.}\PY{n}{GaussianMixture}\PY{p}{(}\PY{n}{n\PYZus{}components}\PY{o}{=}\PY{l+m+mi}{8}\PY{p}{,} \PY{n}{covariance\PYZus{}type}\PY{o}{=}\PY{l+s+s2}{\PYZdq{}}\PY{l+s+s2}{diag}\PY{l+s+s2}{\PYZdq{}}\PY{p}{)}
         \PY{n}{gaussian\PYZus{}fit}\PY{o}{.}\PY{n}{fit}\PY{p}{(}\PY{n}{data}\PY{p}{)}
         
         \PY{n}{plot\PYZus{}mix}\PY{p}{(}\PY{n}{gaussian\PYZus{}fit}\PY{p}{)}
\end{Verbatim}


    \begin{center}
    \adjustimage{max size={0.9\linewidth}{0.9\paperheight}}{output_13_0.png}
    \end{center}
    { \hspace*{\fill} \\}
    

    % Add a bibliography block to the postdoc
    
    
    
    \end{document}
