
% Default to the notebook output style

    


% Inherit from the specified cell style.




    
\documentclass[11pt]{article}

    
    
    \usepackage[T1]{fontenc}
    % Nicer default font (+ math font) than Computer Modern for most use cases
    \usepackage{mathpazo}

    % Basic figure setup, for now with no caption control since it's done
    % automatically by Pandoc (which extracts ![](path) syntax from Markdown).
    \usepackage{graphicx}
    % We will generate all images so they have a width \maxwidth. This means
    % that they will get their normal width if they fit onto the page, but
    % are scaled down if they would overflow the margins.
    \makeatletter
    \def\maxwidth{\ifdim\Gin@nat@width>\linewidth\linewidth
    \else\Gin@nat@width\fi}
    \makeatother
    \let\Oldincludegraphics\includegraphics
    % Set max figure width to be 80% of text width, for now hardcoded.
    \renewcommand{\includegraphics}[1]{\Oldincludegraphics[width=.8\maxwidth]{#1}}
    % Ensure that by default, figures have no caption (until we provide a
    % proper Figure object with a Caption API and a way to capture that
    % in the conversion process - todo).
    \usepackage{caption}
    \DeclareCaptionLabelFormat{nolabel}{}
    \captionsetup{labelformat=nolabel}

    \usepackage{adjustbox} % Used to constrain images to a maximum size 
    \usepackage{xcolor} % Allow colors to be defined
    \usepackage{enumerate} % Needed for markdown enumerations to work
    \usepackage{geometry} % Used to adjust the document margins
    \usepackage{amsmath} % Equations
    \usepackage{amssymb} % Equations
    \usepackage{textcomp} % defines textquotesingle
    % Hack from http://tex.stackexchange.com/a/47451/13684:
    \AtBeginDocument{%
        \def\PYZsq{\textquotesingle}% Upright quotes in Pygmentized code
    }
    \usepackage{upquote} % Upright quotes for verbatim code
    \usepackage{eurosym} % defines \euro
    \usepackage[mathletters]{ucs} % Extended unicode (utf-8) support
    \usepackage[utf8x]{inputenc} % Allow utf-8 characters in the tex document
    \usepackage{fancyvrb} % verbatim replacement that allows latex
    \usepackage{grffile} % extends the file name processing of package graphics 
                         % to support a larger range 
    % The hyperref package gives us a pdf with properly built
    % internal navigation ('pdf bookmarks' for the table of contents,
    % internal cross-reference links, web links for URLs, etc.)
    \usepackage{hyperref}
    \usepackage{longtable} % longtable support required by pandoc >1.10
    \usepackage{booktabs}  % table support for pandoc > 1.12.2
    \usepackage[inline]{enumitem} % IRkernel/repr support (it uses the enumerate* environment)
    \usepackage[normalem]{ulem} % ulem is needed to support strikethroughs (\sout)
                                % normalem makes italics be italics, not underlines
    \usepackage{mathrsfs}
    

    
    
    % Colors for the hyperref package
    \definecolor{urlcolor}{rgb}{0,.145,.698}
    \definecolor{linkcolor}{rgb}{.71,0.21,0.01}
    \definecolor{citecolor}{rgb}{.12,.54,.11}

    % ANSI colors
    \definecolor{ansi-black}{HTML}{3E424D}
    \definecolor{ansi-black-intense}{HTML}{282C36}
    \definecolor{ansi-red}{HTML}{E75C58}
    \definecolor{ansi-red-intense}{HTML}{B22B31}
    \definecolor{ansi-green}{HTML}{00A250}
    \definecolor{ansi-green-intense}{HTML}{007427}
    \definecolor{ansi-yellow}{HTML}{DDB62B}
    \definecolor{ansi-yellow-intense}{HTML}{B27D12}
    \definecolor{ansi-blue}{HTML}{208FFB}
    \definecolor{ansi-blue-intense}{HTML}{0065CA}
    \definecolor{ansi-magenta}{HTML}{D160C4}
    \definecolor{ansi-magenta-intense}{HTML}{A03196}
    \definecolor{ansi-cyan}{HTML}{60C6C8}
    \definecolor{ansi-cyan-intense}{HTML}{258F8F}
    \definecolor{ansi-white}{HTML}{C5C1B4}
    \definecolor{ansi-white-intense}{HTML}{A1A6B2}
    \definecolor{ansi-default-inverse-fg}{HTML}{FFFFFF}
    \definecolor{ansi-default-inverse-bg}{HTML}{000000}

    % commands and environments needed by pandoc snippets
    % extracted from the output of `pandoc -s`
    \providecommand{\tightlist}{%
      \setlength{\itemsep}{0pt}\setlength{\parskip}{0pt}}
    \DefineVerbatimEnvironment{Highlighting}{Verbatim}{commandchars=\\\{\}}
    % Add ',fontsize=\small' for more characters per line
    \newenvironment{Shaded}{}{}
    \newcommand{\KeywordTok}[1]{\textcolor[rgb]{0.00,0.44,0.13}{\textbf{{#1}}}}
    \newcommand{\DataTypeTok}[1]{\textcolor[rgb]{0.56,0.13,0.00}{{#1}}}
    \newcommand{\DecValTok}[1]{\textcolor[rgb]{0.25,0.63,0.44}{{#1}}}
    \newcommand{\BaseNTok}[1]{\textcolor[rgb]{0.25,0.63,0.44}{{#1}}}
    \newcommand{\FloatTok}[1]{\textcolor[rgb]{0.25,0.63,0.44}{{#1}}}
    \newcommand{\CharTok}[1]{\textcolor[rgb]{0.25,0.44,0.63}{{#1}}}
    \newcommand{\StringTok}[1]{\textcolor[rgb]{0.25,0.44,0.63}{{#1}}}
    \newcommand{\CommentTok}[1]{\textcolor[rgb]{0.38,0.63,0.69}{\textit{{#1}}}}
    \newcommand{\OtherTok}[1]{\textcolor[rgb]{0.00,0.44,0.13}{{#1}}}
    \newcommand{\AlertTok}[1]{\textcolor[rgb]{1.00,0.00,0.00}{\textbf{{#1}}}}
    \newcommand{\FunctionTok}[1]{\textcolor[rgb]{0.02,0.16,0.49}{{#1}}}
    \newcommand{\RegionMarkerTok}[1]{{#1}}
    \newcommand{\ErrorTok}[1]{\textcolor[rgb]{1.00,0.00,0.00}{\textbf{{#1}}}}
    \newcommand{\NormalTok}[1]{{#1}}
    
    % Additional commands for more recent versions of Pandoc
    \newcommand{\ConstantTok}[1]{\textcolor[rgb]{0.53,0.00,0.00}{{#1}}}
    \newcommand{\SpecialCharTok}[1]{\textcolor[rgb]{0.25,0.44,0.63}{{#1}}}
    \newcommand{\VerbatimStringTok}[1]{\textcolor[rgb]{0.25,0.44,0.63}{{#1}}}
    \newcommand{\SpecialStringTok}[1]{\textcolor[rgb]{0.73,0.40,0.53}{{#1}}}
    \newcommand{\ImportTok}[1]{{#1}}
    \newcommand{\DocumentationTok}[1]{\textcolor[rgb]{0.73,0.13,0.13}{\textit{{#1}}}}
    \newcommand{\AnnotationTok}[1]{\textcolor[rgb]{0.38,0.63,0.69}{\textbf{\textit{{#1}}}}}
    \newcommand{\CommentVarTok}[1]{\textcolor[rgb]{0.38,0.63,0.69}{\textbf{\textit{{#1}}}}}
    \newcommand{\VariableTok}[1]{\textcolor[rgb]{0.10,0.09,0.49}{{#1}}}
    \newcommand{\ControlFlowTok}[1]{\textcolor[rgb]{0.00,0.44,0.13}{\textbf{{#1}}}}
    \newcommand{\OperatorTok}[1]{\textcolor[rgb]{0.40,0.40,0.40}{{#1}}}
    \newcommand{\BuiltInTok}[1]{{#1}}
    \newcommand{\ExtensionTok}[1]{{#1}}
    \newcommand{\PreprocessorTok}[1]{\textcolor[rgb]{0.74,0.48,0.00}{{#1}}}
    \newcommand{\AttributeTok}[1]{\textcolor[rgb]{0.49,0.56,0.16}{{#1}}}
    \newcommand{\InformationTok}[1]{\textcolor[rgb]{0.38,0.63,0.69}{\textbf{\textit{{#1}}}}}
    \newcommand{\WarningTok}[1]{\textcolor[rgb]{0.38,0.63,0.69}{\textbf{\textit{{#1}}}}}
    
    
    % Define a nice break command that doesn't care if a line doesn't already
    % exist.
    \def\br{\hspace*{\fill} \\* }
    % Math Jax compatibility definitions
    \def\gt{>}
    \def\lt{<}
    \let\Oldtex\TeX
    \let\Oldlatex\LaTeX
    \renewcommand{\TeX}{\textrm{\Oldtex}}
    \renewcommand{\LaTeX}{\textrm{\Oldlatex}}
    % Document parameters
    % Document title
    \title{bojanits\_hwk03}
    
    
    
    
    

    % Pygments definitions
    
\makeatletter
\def\PY@reset{\let\PY@it=\relax \let\PY@bf=\relax%
    \let\PY@ul=\relax \let\PY@tc=\relax%
    \let\PY@bc=\relax \let\PY@ff=\relax}
\def\PY@tok#1{\csname PY@tok@#1\endcsname}
\def\PY@toks#1+{\ifx\relax#1\empty\else%
    \PY@tok{#1}\expandafter\PY@toks\fi}
\def\PY@do#1{\PY@bc{\PY@tc{\PY@ul{%
    \PY@it{\PY@bf{\PY@ff{#1}}}}}}}
\def\PY#1#2{\PY@reset\PY@toks#1+\relax+\PY@do{#2}}

\expandafter\def\csname PY@tok@w\endcsname{\def\PY@tc##1{\textcolor[rgb]{0.73,0.73,0.73}{##1}}}
\expandafter\def\csname PY@tok@c\endcsname{\let\PY@it=\textit\def\PY@tc##1{\textcolor[rgb]{0.25,0.50,0.50}{##1}}}
\expandafter\def\csname PY@tok@cp\endcsname{\def\PY@tc##1{\textcolor[rgb]{0.74,0.48,0.00}{##1}}}
\expandafter\def\csname PY@tok@k\endcsname{\let\PY@bf=\textbf\def\PY@tc##1{\textcolor[rgb]{0.00,0.50,0.00}{##1}}}
\expandafter\def\csname PY@tok@kp\endcsname{\def\PY@tc##1{\textcolor[rgb]{0.00,0.50,0.00}{##1}}}
\expandafter\def\csname PY@tok@kt\endcsname{\def\PY@tc##1{\textcolor[rgb]{0.69,0.00,0.25}{##1}}}
\expandafter\def\csname PY@tok@o\endcsname{\def\PY@tc##1{\textcolor[rgb]{0.40,0.40,0.40}{##1}}}
\expandafter\def\csname PY@tok@ow\endcsname{\let\PY@bf=\textbf\def\PY@tc##1{\textcolor[rgb]{0.67,0.13,1.00}{##1}}}
\expandafter\def\csname PY@tok@nb\endcsname{\def\PY@tc##1{\textcolor[rgb]{0.00,0.50,0.00}{##1}}}
\expandafter\def\csname PY@tok@nf\endcsname{\def\PY@tc##1{\textcolor[rgb]{0.00,0.00,1.00}{##1}}}
\expandafter\def\csname PY@tok@nc\endcsname{\let\PY@bf=\textbf\def\PY@tc##1{\textcolor[rgb]{0.00,0.00,1.00}{##1}}}
\expandafter\def\csname PY@tok@nn\endcsname{\let\PY@bf=\textbf\def\PY@tc##1{\textcolor[rgb]{0.00,0.00,1.00}{##1}}}
\expandafter\def\csname PY@tok@ne\endcsname{\let\PY@bf=\textbf\def\PY@tc##1{\textcolor[rgb]{0.82,0.25,0.23}{##1}}}
\expandafter\def\csname PY@tok@nv\endcsname{\def\PY@tc##1{\textcolor[rgb]{0.10,0.09,0.49}{##1}}}
\expandafter\def\csname PY@tok@no\endcsname{\def\PY@tc##1{\textcolor[rgb]{0.53,0.00,0.00}{##1}}}
\expandafter\def\csname PY@tok@nl\endcsname{\def\PY@tc##1{\textcolor[rgb]{0.63,0.63,0.00}{##1}}}
\expandafter\def\csname PY@tok@ni\endcsname{\let\PY@bf=\textbf\def\PY@tc##1{\textcolor[rgb]{0.60,0.60,0.60}{##1}}}
\expandafter\def\csname PY@tok@na\endcsname{\def\PY@tc##1{\textcolor[rgb]{0.49,0.56,0.16}{##1}}}
\expandafter\def\csname PY@tok@nt\endcsname{\let\PY@bf=\textbf\def\PY@tc##1{\textcolor[rgb]{0.00,0.50,0.00}{##1}}}
\expandafter\def\csname PY@tok@nd\endcsname{\def\PY@tc##1{\textcolor[rgb]{0.67,0.13,1.00}{##1}}}
\expandafter\def\csname PY@tok@s\endcsname{\def\PY@tc##1{\textcolor[rgb]{0.73,0.13,0.13}{##1}}}
\expandafter\def\csname PY@tok@sd\endcsname{\let\PY@it=\textit\def\PY@tc##1{\textcolor[rgb]{0.73,0.13,0.13}{##1}}}
\expandafter\def\csname PY@tok@si\endcsname{\let\PY@bf=\textbf\def\PY@tc##1{\textcolor[rgb]{0.73,0.40,0.53}{##1}}}
\expandafter\def\csname PY@tok@se\endcsname{\let\PY@bf=\textbf\def\PY@tc##1{\textcolor[rgb]{0.73,0.40,0.13}{##1}}}
\expandafter\def\csname PY@tok@sr\endcsname{\def\PY@tc##1{\textcolor[rgb]{0.73,0.40,0.53}{##1}}}
\expandafter\def\csname PY@tok@ss\endcsname{\def\PY@tc##1{\textcolor[rgb]{0.10,0.09,0.49}{##1}}}
\expandafter\def\csname PY@tok@sx\endcsname{\def\PY@tc##1{\textcolor[rgb]{0.00,0.50,0.00}{##1}}}
\expandafter\def\csname PY@tok@m\endcsname{\def\PY@tc##1{\textcolor[rgb]{0.40,0.40,0.40}{##1}}}
\expandafter\def\csname PY@tok@gh\endcsname{\let\PY@bf=\textbf\def\PY@tc##1{\textcolor[rgb]{0.00,0.00,0.50}{##1}}}
\expandafter\def\csname PY@tok@gu\endcsname{\let\PY@bf=\textbf\def\PY@tc##1{\textcolor[rgb]{0.50,0.00,0.50}{##1}}}
\expandafter\def\csname PY@tok@gd\endcsname{\def\PY@tc##1{\textcolor[rgb]{0.63,0.00,0.00}{##1}}}
\expandafter\def\csname PY@tok@gi\endcsname{\def\PY@tc##1{\textcolor[rgb]{0.00,0.63,0.00}{##1}}}
\expandafter\def\csname PY@tok@gr\endcsname{\def\PY@tc##1{\textcolor[rgb]{1.00,0.00,0.00}{##1}}}
\expandafter\def\csname PY@tok@ge\endcsname{\let\PY@it=\textit}
\expandafter\def\csname PY@tok@gs\endcsname{\let\PY@bf=\textbf}
\expandafter\def\csname PY@tok@gp\endcsname{\let\PY@bf=\textbf\def\PY@tc##1{\textcolor[rgb]{0.00,0.00,0.50}{##1}}}
\expandafter\def\csname PY@tok@go\endcsname{\def\PY@tc##1{\textcolor[rgb]{0.53,0.53,0.53}{##1}}}
\expandafter\def\csname PY@tok@gt\endcsname{\def\PY@tc##1{\textcolor[rgb]{0.00,0.27,0.87}{##1}}}
\expandafter\def\csname PY@tok@err\endcsname{\def\PY@bc##1{\setlength{\fboxsep}{0pt}\fcolorbox[rgb]{1.00,0.00,0.00}{1,1,1}{\strut ##1}}}
\expandafter\def\csname PY@tok@kc\endcsname{\let\PY@bf=\textbf\def\PY@tc##1{\textcolor[rgb]{0.00,0.50,0.00}{##1}}}
\expandafter\def\csname PY@tok@kd\endcsname{\let\PY@bf=\textbf\def\PY@tc##1{\textcolor[rgb]{0.00,0.50,0.00}{##1}}}
\expandafter\def\csname PY@tok@kn\endcsname{\let\PY@bf=\textbf\def\PY@tc##1{\textcolor[rgb]{0.00,0.50,0.00}{##1}}}
\expandafter\def\csname PY@tok@kr\endcsname{\let\PY@bf=\textbf\def\PY@tc##1{\textcolor[rgb]{0.00,0.50,0.00}{##1}}}
\expandafter\def\csname PY@tok@bp\endcsname{\def\PY@tc##1{\textcolor[rgb]{0.00,0.50,0.00}{##1}}}
\expandafter\def\csname PY@tok@fm\endcsname{\def\PY@tc##1{\textcolor[rgb]{0.00,0.00,1.00}{##1}}}
\expandafter\def\csname PY@tok@vc\endcsname{\def\PY@tc##1{\textcolor[rgb]{0.10,0.09,0.49}{##1}}}
\expandafter\def\csname PY@tok@vg\endcsname{\def\PY@tc##1{\textcolor[rgb]{0.10,0.09,0.49}{##1}}}
\expandafter\def\csname PY@tok@vi\endcsname{\def\PY@tc##1{\textcolor[rgb]{0.10,0.09,0.49}{##1}}}
\expandafter\def\csname PY@tok@vm\endcsname{\def\PY@tc##1{\textcolor[rgb]{0.10,0.09,0.49}{##1}}}
\expandafter\def\csname PY@tok@sa\endcsname{\def\PY@tc##1{\textcolor[rgb]{0.73,0.13,0.13}{##1}}}
\expandafter\def\csname PY@tok@sb\endcsname{\def\PY@tc##1{\textcolor[rgb]{0.73,0.13,0.13}{##1}}}
\expandafter\def\csname PY@tok@sc\endcsname{\def\PY@tc##1{\textcolor[rgb]{0.73,0.13,0.13}{##1}}}
\expandafter\def\csname PY@tok@dl\endcsname{\def\PY@tc##1{\textcolor[rgb]{0.73,0.13,0.13}{##1}}}
\expandafter\def\csname PY@tok@s2\endcsname{\def\PY@tc##1{\textcolor[rgb]{0.73,0.13,0.13}{##1}}}
\expandafter\def\csname PY@tok@sh\endcsname{\def\PY@tc##1{\textcolor[rgb]{0.73,0.13,0.13}{##1}}}
\expandafter\def\csname PY@tok@s1\endcsname{\def\PY@tc##1{\textcolor[rgb]{0.73,0.13,0.13}{##1}}}
\expandafter\def\csname PY@tok@mb\endcsname{\def\PY@tc##1{\textcolor[rgb]{0.40,0.40,0.40}{##1}}}
\expandafter\def\csname PY@tok@mf\endcsname{\def\PY@tc##1{\textcolor[rgb]{0.40,0.40,0.40}{##1}}}
\expandafter\def\csname PY@tok@mh\endcsname{\def\PY@tc##1{\textcolor[rgb]{0.40,0.40,0.40}{##1}}}
\expandafter\def\csname PY@tok@mi\endcsname{\def\PY@tc##1{\textcolor[rgb]{0.40,0.40,0.40}{##1}}}
\expandafter\def\csname PY@tok@il\endcsname{\def\PY@tc##1{\textcolor[rgb]{0.40,0.40,0.40}{##1}}}
\expandafter\def\csname PY@tok@mo\endcsname{\def\PY@tc##1{\textcolor[rgb]{0.40,0.40,0.40}{##1}}}
\expandafter\def\csname PY@tok@ch\endcsname{\let\PY@it=\textit\def\PY@tc##1{\textcolor[rgb]{0.25,0.50,0.50}{##1}}}
\expandafter\def\csname PY@tok@cm\endcsname{\let\PY@it=\textit\def\PY@tc##1{\textcolor[rgb]{0.25,0.50,0.50}{##1}}}
\expandafter\def\csname PY@tok@cpf\endcsname{\let\PY@it=\textit\def\PY@tc##1{\textcolor[rgb]{0.25,0.50,0.50}{##1}}}
\expandafter\def\csname PY@tok@c1\endcsname{\let\PY@it=\textit\def\PY@tc##1{\textcolor[rgb]{0.25,0.50,0.50}{##1}}}
\expandafter\def\csname PY@tok@cs\endcsname{\let\PY@it=\textit\def\PY@tc##1{\textcolor[rgb]{0.25,0.50,0.50}{##1}}}

\def\PYZbs{\char`\\}
\def\PYZus{\char`\_}
\def\PYZob{\char`\{}
\def\PYZcb{\char`\}}
\def\PYZca{\char`\^}
\def\PYZam{\char`\&}
\def\PYZlt{\char`\<}
\def\PYZgt{\char`\>}
\def\PYZsh{\char`\#}
\def\PYZpc{\char`\%}
\def\PYZdl{\char`\$}
\def\PYZhy{\char`\-}
\def\PYZsq{\char`\'}
\def\PYZdq{\char`\"}
\def\PYZti{\char`\~}
% for compatibility with earlier versions
\def\PYZat{@}
\def\PYZlb{[}
\def\PYZrb{]}
\makeatother


    % Exact colors from NB
    \definecolor{incolor}{rgb}{0.0, 0.0, 0.5}
    \definecolor{outcolor}{rgb}{0.545, 0.0, 0.0}



    
    % Prevent overflowing lines due to hard-to-break entities
    \sloppy 
    % Setup hyperref package
    \hypersetup{
      breaklinks=true,  % so long urls are correctly broken across lines
      colorlinks=true,
      urlcolor=urlcolor,
      linkcolor=linkcolor,
      citecolor=citecolor,
      }
    % Slightly bigger margins than the latex defaults
    
    \geometry{verbose,tmargin=1in,bmargin=1in,lmargin=1in,rmargin=1in}
    
    

    \begin{document}
    
    
    \maketitle
    
    

    
    \begin{Verbatim}[commandchars=\\\{\}]
{\color{incolor}In [{\color{incolor}130}]:} \PY{o}{\PYZpc{}}\PY{k}{pylab} inline
\end{Verbatim}

    \begin{Verbatim}[commandchars=\\\{\}]
Populating the interactive namespace from numpy and matplotlib

    \end{Verbatim}

    \section{Name: Tommy Bojanin}\label{name-tommy-bojanin}

    \section{Instructions}\label{instructions}

\begin{itemize}
\tightlist
\item
  For math problems, you are \emph{\textbf{encouraged}} to use
  \(\LaTeX,\) however handwritten solutions are also accepted provided
  that they are \textbf{extremely} tidy and thorough.
\item
  Please make sure you explain the rationale behind each step -\/- this
  is very important because:
\end{itemize}

\begin{enumerate}
\def\labelenumi{\arabic{enumi}.}
\tightlist
\item
  It shows me that you understood each step and that you are not (just)
  copying from a friend or a solution found online.
\item
  It helps me understand your approach, I often see you using new
  approaches that are correct and also different that what I or the
  author came up with.
\end{enumerate}

\begin{itemize}
\item
   \textbf{ALSO:}, please \textbf{staple} your solutions before
  submitting them. I will start deducting points for unstapled
  submissions if this continues to be a problem. Be a good friend and
  offer to share a stapler with a freind if they need it -\/- I cannot
  keep track of a stack of loose paper!
\item
  Pleas make sure all of your code fits on the printed pages, this may
  require breaking long lines into several shorter ones. I must see your
  code in order to beleive it is correct.
\end{itemize}

    \section{Problems}\label{problems}

    \subsection{Problem 1}\label{problem-1}

4.14 (*) Show that for a linearly separable data set, the maximum
likelihood solution for the logistic regression model is obtained by
finding a vector \(\mathbf{w}\) whose decision boundary
\(\mathbf{w}^T \phi(\mathbf{x}) = 0\) separates the classes and then
taking the magnitude of \(\mathbf{w}\) to infinity.\\
\textgreater{} \textbf{NOTE:} is this a good or a bad thing about the
maximum likelihood solution...?

    \subsection{Problem 2}\label{problem-2}

4.18 (*) Using the result (4.91) for the derivatives of the softmax
activation function, show that the gradients of the cross-entropy error
(4.108) are given by (4.109).

    \section{Problem 3}\label{problem-3}

Demonstrate the
\href{http://scikit-learn.org/stable/modules/generated/sklearn.linear_model.LogisticRegression.html\#sklearn.linear_model.LogisticRegression}{LogisticRegression}
class from sklearn on a \emph{two dimensonal} slice of either: 1. The
fisher iris data 2. Data from one of the kaggle competitions

You must: 1. Separate your data into five different \emph{test} and
\emph{training} sets, using
\href{http://scikit-learn.org/stable/modules/generated/sklearn.model_selection.StratifiedKFold.html\#sklearn.model_selection.StratifiedKFold}{Stratified
Kfold} cross validation, making sure that you set the parameter to
shuffle your data. 1. For each of the 5 different folds/classifiers: 1.
Show the the way the space is partitioned, e.g. Fig 2.28 or Fig 4.5. 1.
Include a scatter plot of the training data, using colors to indicate
the different target labels. 1. Include a scatter plot of the test data
using a different marker (e.g. squares), also using different colors to
indicate expected labels. 1. Use the
\href{http://scikit-learn.org/stable/modules/model_evaluation.html\#classification-report}{Classification
Report} and the
\href{http://scikit-learn.org/stable/modules/model_evaluation.html\#confusion-matrix}{Confusion
Matrix} to present the quality of your classifiers.

You must share code as well as figures. Comment on how well you think
your classifier will work on a secret test set (e.g. Kaggle's test set)

    \begin{quote}
\textbf{HINT:} I do not expect this to take a large amount of code; most
of what I ask is available already in scipy / sklearn
\end{quote}

    I did this in about 30 minutes and 35 lines of code (other than what I
have shared with you). Here is the output \emph{I} produced for the
first fold. I made use of the \texttt{plot\_confusion\_matrix} function
that I have included in this notebook for you to use as well.

\subsubsection{Fold \# 1 (of 5)}\label{fold-1-of-5}

\begin{verbatim}
             precision    recall  f1-score   support

     setosa       1.00      1.00      1.00        10
 versicolor       1.00      0.70      0.82        10
  virginica       0.77      1.00      0.87        10

avg / total       0.92      0.90      0.90        30
\end{verbatim}

    \begin{Verbatim}[commandchars=\\\{\}]
{\color{incolor}In [{\color{incolor}131}]:} \PY{k+kn}{from} \PY{n+nn}{sklearn}\PY{n+nn}{.}\PY{n+nn}{datasets} \PY{k}{import} \PY{n}{load\PYZus{}iris}  \PY{c+c1}{\PYZsh{} or use some other data, that is ok}
          \PY{k+kn}{from} \PY{n+nn}{sklearn}\PY{n+nn}{.}\PY{n+nn}{linear\PYZus{}model} \PY{k}{import} \PY{n}{LogisticRegression}
          \PY{k+kn}{from} \PY{n+nn}{sklearn}\PY{n+nn}{.}\PY{n+nn}{model\PYZus{}selection} \PY{k}{import} \PY{n}{StratifiedKFold}
          \PY{k+kn}{from} \PY{n+nn}{sklearn}\PY{n+nn}{.}\PY{n+nn}{metrics} \PY{k}{import} \PY{n}{confusion\PYZus{}matrix}\PY{p}{,} \PY{n}{classification\PYZus{}report}
          \PY{k+kn}{from} \PY{n+nn}{matplotlib} \PY{k}{import} \PY{n}{pyplot} \PY{k}{as} \PY{n}{plt}
\end{Verbatim}

    \begin{Verbatim}[commandchars=\\\{\}]
{\color{incolor}In [{\color{incolor}132}]:} \PY{k+kn}{import} \PY{n+nn}{itertools}
          \PY{k}{def} \PY{n+nf}{plot\PYZus{}confusion\PYZus{}matrix}\PY{p}{(}\PY{n}{cm}\PY{p}{,} \PY{n}{classes}\PY{p}{,}
                                    \PY{n}{normalize}\PY{o}{=}\PY{k+kc}{False}\PY{p}{,}
                                    \PY{n}{title}\PY{o}{=}\PY{l+s+s1}{\PYZsq{}}\PY{l+s+s1}{Confusion matrix}\PY{l+s+s1}{\PYZsq{}}\PY{p}{,}
                                    \PY{n}{cmap}\PY{o}{=}\PY{n}{plt}\PY{o}{.}\PY{n}{cm}\PY{o}{.}\PY{n}{Blues}\PY{p}{)}\PY{p}{:}
              \PY{l+s+sd}{\PYZdq{}\PYZdq{}\PYZdq{}}
          \PY{l+s+sd}{    This function prints and plots the confusion matrix.}
          \PY{l+s+sd}{    Normalization can be applied by setting `normalize=True`.}
          \PY{l+s+sd}{    \PYZdq{}\PYZdq{}\PYZdq{}}
              \PY{k}{if} \PY{n}{normalize}\PY{p}{:}
                  \PY{n}{cm} \PY{o}{=} \PY{n}{cm}\PY{o}{.}\PY{n}{astype}\PY{p}{(}\PY{l+s+s1}{\PYZsq{}}\PY{l+s+s1}{float}\PY{l+s+s1}{\PYZsq{}}\PY{p}{)} \PY{o}{/} \PY{n}{cm}\PY{o}{.}\PY{n}{sum}\PY{p}{(}\PY{n}{axis}\PY{o}{=}\PY{l+m+mi}{1}\PY{p}{)}\PY{p}{[}\PY{p}{:}\PY{p}{,} \PY{n}{np}\PY{o}{.}\PY{n}{newaxis}\PY{p}{]}
          
              \PY{n}{plt}\PY{o}{.}\PY{n}{imshow}\PY{p}{(}\PY{n}{cm}\PY{p}{,} \PY{n}{interpolation}\PY{o}{=}\PY{l+s+s1}{\PYZsq{}}\PY{l+s+s1}{nearest}\PY{l+s+s1}{\PYZsq{}}\PY{p}{,} \PY{n}{cmap}\PY{o}{=}\PY{n}{cmap}\PY{p}{)}
              \PY{n}{plt}\PY{o}{.}\PY{n}{title}\PY{p}{(}\PY{n}{title}\PY{p}{)}
              \PY{n}{plt}\PY{o}{.}\PY{n}{colorbar}\PY{p}{(}\PY{p}{)}
              \PY{n}{tick\PYZus{}marks} \PY{o}{=} \PY{n}{np}\PY{o}{.}\PY{n}{arange}\PY{p}{(}\PY{n+nb}{len}\PY{p}{(}\PY{n}{classes}\PY{p}{)}\PY{p}{)}
              \PY{n}{plt}\PY{o}{.}\PY{n}{xticks}\PY{p}{(}\PY{n}{tick\PYZus{}marks}\PY{p}{,} \PY{n}{classes}\PY{p}{,} \PY{n}{rotation}\PY{o}{=}\PY{l+m+mi}{45}\PY{p}{)}
              \PY{n}{plt}\PY{o}{.}\PY{n}{yticks}\PY{p}{(}\PY{n}{tick\PYZus{}marks}\PY{p}{,} \PY{n}{classes}\PY{p}{)}
          
              \PY{n}{fmt} \PY{o}{=} \PY{l+s+s1}{\PYZsq{}}\PY{l+s+s1}{.2f}\PY{l+s+s1}{\PYZsq{}} \PY{k}{if} \PY{n}{normalize} \PY{k}{else} \PY{l+s+s1}{\PYZsq{}}\PY{l+s+s1}{d}\PY{l+s+s1}{\PYZsq{}}
              \PY{n}{thresh} \PY{o}{=} \PY{n}{cm}\PY{o}{.}\PY{n}{max}\PY{p}{(}\PY{p}{)} \PY{o}{/} \PY{l+m+mf}{2.}
              \PY{k}{for} \PY{n}{i}\PY{p}{,} \PY{n}{j} \PY{o+ow}{in} \PY{n}{itertools}\PY{o}{.}\PY{n}{product}\PY{p}{(}\PY{n+nb}{range}\PY{p}{(}\PY{n}{cm}\PY{o}{.}\PY{n}{shape}\PY{p}{[}\PY{l+m+mi}{0}\PY{p}{]}\PY{p}{)}\PY{p}{,} \PY{n+nb}{range}\PY{p}{(}\PY{n}{cm}\PY{o}{.}\PY{n}{shape}\PY{p}{[}\PY{l+m+mi}{1}\PY{p}{]}\PY{p}{)}\PY{p}{)}\PY{p}{:}
                  \PY{n}{plt}\PY{o}{.}\PY{n}{text}\PY{p}{(}\PY{n}{j}\PY{p}{,} \PY{n}{i}\PY{p}{,} \PY{n+nb}{format}\PY{p}{(}\PY{n}{cm}\PY{p}{[}\PY{n}{i}\PY{p}{,} \PY{n}{j}\PY{p}{]}\PY{p}{,} \PY{n}{fmt}\PY{p}{)}\PY{p}{,}
                           \PY{n}{horizontalalignment}\PY{o}{=}\PY{l+s+s2}{\PYZdq{}}\PY{l+s+s2}{center}\PY{l+s+s2}{\PYZdq{}}\PY{p}{,}
                           \PY{n}{color}\PY{o}{=}\PY{l+s+s2}{\PYZdq{}}\PY{l+s+s2}{white}\PY{l+s+s2}{\PYZdq{}} \PY{k}{if} \PY{n}{cm}\PY{p}{[}\PY{n}{i}\PY{p}{,} \PY{n}{j}\PY{p}{]} \PY{o}{\PYZgt{}} \PY{n}{thresh} \PY{k}{else} \PY{l+s+s2}{\PYZdq{}}\PY{l+s+s2}{black}\PY{l+s+s2}{\PYZdq{}}\PY{p}{)}
          
              \PY{n}{plt}\PY{o}{.}\PY{n}{tight\PYZus{}layout}\PY{p}{(}\PY{p}{)}
              \PY{n}{plt}\PY{o}{.}\PY{n}{ylabel}\PY{p}{(}\PY{l+s+s1}{\PYZsq{}}\PY{l+s+s1}{True label}\PY{l+s+s1}{\PYZsq{}}\PY{p}{)}
              \PY{n}{plt}\PY{o}{.}\PY{n}{xlabel}\PY{p}{(}\PY{l+s+s1}{\PYZsq{}}\PY{l+s+s1}{Predicted label}\PY{l+s+s1}{\PYZsq{}}\PY{p}{)}
\end{Verbatim}

    \begin{Verbatim}[commandchars=\\\{\}]
{\color{incolor}In [{\color{incolor}155}]:} \PY{n}{iris} \PY{o}{=} \PY{n}{load\PYZus{}iris}\PY{p}{(}\PY{p}{)}
          \PY{n}{X}\PY{p}{,}\PY{n}{y} \PY{o}{=} \PY{n}{iris}\PY{o}{.}\PY{n}{data}\PY{p}{,} \PY{n}{iris}\PY{o}{.}\PY{n}{target}
          \PY{n}{target\PYZus{}names} \PY{o}{=} \PY{n}{iris}\PY{o}{.}\PY{n}{target\PYZus{}names}
          
          
              
          
              
          \PY{n}{kfold} \PY{o}{=} \PY{n}{StratifiedKFold}\PY{p}{(}\PY{n}{n\PYZus{}splits} \PY{o}{=} \PY{l+m+mi}{5}\PY{p}{,} \PY{n}{shuffle} \PY{o}{=} \PY{k+kc}{True}\PY{p}{)}
          
          \PY{k}{for} \PY{n}{train\PYZus{}index}\PY{p}{,} \PY{n}{test\PYZus{}index} \PY{o+ow}{in} \PY{n}{kfold}\PY{o}{.}\PY{n}{split}\PY{p}{(}\PY{n}{X}\PY{p}{,}\PY{n}{y}\PY{p}{)}\PY{p}{:}
              
              \PY{c+c1}{\PYZsh{}create training and test sets}
              \PY{n}{X\PYZus{}train}\PY{p}{,} \PY{n}{X\PYZus{}test} \PY{o}{=} \PY{n}{X}\PY{p}{[}\PY{n}{train\PYZus{}index}\PY{p}{]}\PY{p}{,} \PY{n}{X}\PY{p}{[}\PY{n}{test\PYZus{}index}\PY{p}{]}
              \PY{n}{y\PYZus{}train}\PY{p}{,} \PY{n}{y\PYZus{}test} \PY{o}{=} \PY{n}{y}\PY{p}{[}\PY{n}{train\PYZus{}index}\PY{p}{]}\PY{p}{,} \PY{n}{y}\PY{p}{[}\PY{n}{test\PYZus{}index}\PY{p}{]}
              
              \PY{c+c1}{\PYZsh{}add training features}
              \PY{n}{train\PYZus{}featuresAll}\PY{o}{=}\PY{p}{[}\PY{p}{]}
              \PY{n}{train\PYZus{}targets} \PY{o}{=} \PY{p}{[}\PY{p}{]}
              \PY{n}{train\PYZus{}features} \PY{o}{=} \PY{n}{X\PYZus{}train}\PY{p}{[}\PY{p}{:} \PY{p}{,} \PY{p}{[}\PY{l+m+mi}{0}\PY{p}{,}\PY{l+m+mi}{1}\PY{p}{,}\PY{l+m+mi}{2}\PY{p}{,}\PY{l+m+mi}{3}\PY{p}{]}\PY{p}{]}
              
              \PY{k}{for} \PY{n}{feature} \PY{o+ow}{in} \PY{n}{train\PYZus{}features}\PY{p}{:}
                  \PY{n}{train\PYZus{}featuresAll}\PY{o}{.}\PY{n}{append}\PY{p}{(}\PY{n}{feature}\PY{p}{[}\PY{l+m+mi}{0}\PY{p}{]}\PY{p}{)} 
                  \PY{n}{train\PYZus{}targets}\PY{o}{.}\PY{n}{append}\PY{p}{(}\PY{n}{feature}\PY{p}{[}\PY{l+m+mi}{1}\PY{p}{]}\PY{p}{)}
              
              \PY{n}{train\PYZus{}data} \PY{o}{=} \PY{p}{(}\PY{p}{(}\PY{n}{train\PYZus{}featuresAll}\PY{p}{[}\PY{p}{:}\PY{l+m+mi}{50}\PY{p}{]}\PY{p}{,} \PY{n}{train\PYZus{}targets}\PY{p}{[}\PY{p}{:}\PY{l+m+mi}{50}\PY{p}{]}\PY{p}{)}\PY{p}{,} \PY{p}{(}\PY{n}{train\PYZus{}featuresAll}\PY{p}{[}\PY{l+m+mi}{50}\PY{p}{:}\PY{l+m+mi}{100}\PY{p}{]}\PY{p}{,} \PY{n}{train\PYZus{}targets}\PY{p}{[}\PY{l+m+mi}{50}\PY{p}{:}\PY{l+m+mi}{100}\PY{p}{]}\PY{p}{)}\PY{p}{,} 
                  \PY{p}{(}\PY{n}{train\PYZus{}featuresAll}\PY{p}{[}\PY{l+m+mi}{100}\PY{p}{:}\PY{l+m+mi}{150}\PY{p}{]}\PY{p}{,} \PY{n}{train\PYZus{}targets}\PY{p}{[}\PY{l+m+mi}{100}\PY{p}{:}\PY{l+m+mi}{150}\PY{p}{]}\PY{p}{)}\PY{p}{)}
              
              
              \PY{c+c1}{\PYZsh{}do the same thing for the test data}
              \PY{n}{test\PYZus{}targets} \PY{o}{=} \PY{p}{[}\PY{p}{]}
              \PY{n}{test\PYZus{}features} \PY{o}{=} \PY{n}{X\PYZus{}test}\PY{p}{[}\PY{p}{:}\PY{p}{,} \PY{p}{[}\PY{l+m+mi}{0}\PY{p}{,}\PY{l+m+mi}{1}\PY{p}{,}\PY{l+m+mi}{2}\PY{p}{,}\PY{l+m+mi}{3}\PY{p}{]}\PY{p}{]}
              \PY{n}{test\PYZus{}featuresAll} \PY{o}{=} \PY{p}{[}\PY{p}{]}
              
              \PY{k}{for} \PY{n}{feature} \PY{o+ow}{in} \PY{n}{test\PYZus{}features}\PY{p}{:}
                  \PY{n}{test\PYZus{}featuresAll}\PY{o}{.}\PY{n}{append}\PY{p}{(}\PY{n}{feature}\PY{p}{[}\PY{l+m+mi}{0}\PY{p}{]}\PY{p}{)} 
                  \PY{n}{test\PYZus{}targets}\PY{o}{.}\PY{n}{append}\PY{p}{(}\PY{n}{feature}\PY{p}{[}\PY{l+m+mi}{1}\PY{p}{]}\PY{p}{)}
              
              \PY{n}{test\PYZus{}data} \PY{o}{=} \PY{p}{(}\PY{p}{(}\PY{n}{test\PYZus{}featuresAll}\PY{p}{[}\PY{p}{:}\PY{l+m+mi}{50}\PY{p}{]}\PY{p}{,} \PY{n}{test\PYZus{}targets}\PY{p}{[}\PY{p}{:}\PY{l+m+mi}{50}\PY{p}{]}\PY{p}{)}\PY{p}{,} \PY{p}{(}\PY{n}{test\PYZus{}featuresAll}\PY{p}{[}\PY{l+m+mi}{50}\PY{p}{:}\PY{l+m+mi}{100}\PY{p}{]}\PY{p}{,} \PY{n}{test\PYZus{}targets}\PY{p}{[}\PY{l+m+mi}{50}\PY{p}{:}\PY{l+m+mi}{100}\PY{p}{]}\PY{p}{)}\PY{p}{,} 
                  \PY{p}{(}\PY{n}{test\PYZus{}featuresAll}\PY{p}{[}\PY{l+m+mi}{100}\PY{p}{:}\PY{l+m+mi}{150}\PY{p}{]}\PY{p}{,} \PY{n}{test\PYZus{}targets}\PY{p}{[}\PY{l+m+mi}{100}\PY{p}{:}\PY{l+m+mi}{150}\PY{p}{]}\PY{p}{)}\PY{p}{)}
              
              \PY{c+c1}{\PYZsh{}colors and groups for the scatter plots}
              \PY{n}{groups} \PY{o}{=} \PY{p}{(}\PY{l+s+s1}{\PYZsq{}}\PY{l+s+s1}{Iris\PYZhy{}setosa}\PY{l+s+s1}{\PYZsq{}}\PY{p}{,}\PY{l+s+s1}{\PYZsq{}}\PY{l+s+s1}{Iris\PYZhy{}versicolor}\PY{l+s+s1}{\PYZsq{}}\PY{p}{,}\PY{l+s+s1}{\PYZsq{}}\PY{l+s+s1}{Iris\PYZhy{}virginica}\PY{l+s+s1}{\PYZsq{}}\PY{p}{)}
              \PY{n}{colors} \PY{o}{=} \PY{p}{(}\PY{l+s+s1}{\PYZsq{}}\PY{l+s+s1}{blue}\PY{l+s+s1}{\PYZsq{}}\PY{p}{,} \PY{l+s+s1}{\PYZsq{}}\PY{l+s+s1}{green}\PY{l+s+s1}{\PYZsq{}}\PY{p}{,}\PY{l+s+s1}{\PYZsq{}}\PY{l+s+s1}{red}\PY{l+s+s1}{\PYZsq{}}\PY{p}{)}
              
              
              \PY{c+c1}{\PYZsh{}these two lines mess up my confusion matrix and the classifying report}
              \PY{n}{y\PYZus{}pred} \PY{o}{=} \PY{n}{clf}\PY{o}{.}\PY{n}{fit}\PY{p}{(}\PY{n}{X\PYZus{}train}\PY{p}{,} \PY{n}{y\PYZus{}train}\PY{p}{)}\PY{o}{.}\PY{n}{predict}\PY{p}{(}\PY{n}{X\PYZus{}test}\PY{p}{)}
              \PY{n}{cf\PYZus{}matrix} \PY{o}{=} \PY{n}{confusion\PYZus{}matrix}\PY{p}{(}\PY{n}{y\PYZus{}test}\PY{p}{,} \PY{n}{y\PYZus{}pred}\PY{p}{)}
          
          
                  \PY{c+c1}{\PYZsh{}creating the scatter plots for each iterations (fold)}
              \PY{k}{for} \PY{n}{test\PYZus{}item}\PY{p}{,} \PY{n}{train\PYZus{}item}\PY{p}{,} \PY{n}{color}\PY{p}{,} \PY{n}{group} \PY{o+ow}{in} \PY{n+nb}{zip}\PY{p}{(}\PY{n}{test\PYZus{}data}\PY{p}{,}\PY{n}{train\PYZus{}data}\PY{p}{,} \PY{n}{colors}\PY{p}{,}\PY{n}{groups}\PY{p}{)}\PY{p}{:} 
                  \PY{n}{train\PYZus{}x}\PY{p}{,} \PY{n}{train\PYZus{}y} \PY{o}{=} \PY{n}{train\PYZus{}item}
                  \PY{n}{test\PYZus{}x}\PY{p}{,} \PY{n}{test\PYZus{}y} \PY{o}{=} \PY{n}{test\PYZus{}item}
                  \PY{n}{plt}\PY{o}{.}\PY{n}{scatter}\PY{p}{(}\PY{n}{train\PYZus{}x}\PY{p}{,} \PY{n}{train\PYZus{}y}\PY{p}{,}\PY{n}{color}\PY{o}{=}\PY{n}{color}\PY{p}{,}\PY{n}{alpha}\PY{o}{=}\PY{l+m+mi}{1}\PY{p}{)}
                  \PY{n}{plt}\PY{o}{.}\PY{n}{scatter}\PY{p}{(}\PY{n}{test\PYZus{}x}\PY{p}{,}\PY{n}{test\PYZus{}y}\PY{p}{,}\PY{n}{color}\PY{o}{=}\PY{n}{color}\PY{p}{,}\PY{n}{alpha}\PY{o}{=}\PY{l+m+mi}{1}\PY{p}{,}\PY{n}{marker}\PY{o}{=}\PY{l+s+s1}{\PYZsq{}}\PY{l+s+s1}{s}\PY{l+s+s1}{\PYZsq{}}\PY{p}{)}
                  \PY{n}{plt}\PY{o}{.}\PY{n}{title}\PY{p}{(}\PY{l+s+s1}{\PYZsq{}}\PY{l+s+s1}{Iris scatter plot}\PY{l+s+s1}{\PYZsq{}}\PY{p}{)}
              \PY{n}{plt}\PY{o}{.}\PY{n}{xlabel}\PY{p}{(}\PY{l+s+s1}{\PYZsq{}}\PY{l+s+s1}{sepal length}\PY{l+s+s1}{\PYZsq{}}\PY{p}{)}
              \PY{n}{plt}\PY{o}{.}\PY{n}{ylabel}\PY{p}{(}\PY{l+s+s1}{\PYZsq{}}\PY{l+s+s1}{Sepal width}\PY{l+s+s1}{\PYZsq{}}\PY{p}{)}
              \PY{n}{plt}\PY{o}{.}\PY{n}{show}\PY{p}{(}\PY{p}{)}
              
              \PY{c+c1}{\PYZsh{}printing matrix and classification report}
              \PY{n}{plot\PYZus{}confusion\PYZus{}matrix}\PY{p}{(}\PY{n}{cf\PYZus{}matrix}\PY{p}{,} \PY{n}{classes}\PY{o}{=}\PY{n}{target\PYZus{}names}\PY{p}{,} \PY{n}{title}\PY{o}{=}\PY{l+s+s1}{\PYZsq{}}\PY{l+s+s1}{unnormalized matrix}\PY{l+s+s1}{\PYZsq{}}\PY{p}{)}
              \PY{n}{plt}\PY{o}{.}\PY{n}{show}\PY{p}{(}\PY{p}{)}
              \PY{n}{plot\PYZus{}confusion\PYZus{}matrix}\PY{p}{(}\PY{n}{cf\PYZus{}matrix}\PY{p}{,} \PY{n}{classes}\PY{o}{=}\PY{n}{target\PYZus{}names}\PY{p}{,}\PY{n}{normalize} \PY{o}{=} \PY{k+kc}{True}\PY{p}{,} \PY{n}{title}\PY{o}{=}\PY{l+s+s1}{\PYZsq{}}\PY{l+s+s1}{normalized matrix}\PY{l+s+s1}{\PYZsq{}}\PY{p}{)}
              \PY{n}{plt}\PY{o}{.}\PY{n}{show}\PY{p}{(}\PY{p}{)}
              \PY{n+nb}{print}\PY{p}{(}\PY{n}{classification\PYZus{}report}\PY{p}{(}\PY{n}{y\PYZus{}test}\PY{p}{,} \PY{n}{y\PYZus{}pred}\PY{p}{,} \PY{n}{target\PYZus{}names}\PY{o}{=}\PY{n}{target\PYZus{}names}\PY{p}{)}\PY{p}{)}
              
\end{Verbatim}

    \begin{center}
    \adjustimage{max size={0.9\linewidth}{0.9\paperheight}}{output_11_0.png}
    \end{center}
    { \hspace*{\fill} \\}
    
    \begin{center}
    \adjustimage{max size={0.9\linewidth}{0.9\paperheight}}{output_11_1.png}
    \end{center}
    { \hspace*{\fill} \\}
    
    \begin{center}
    \adjustimage{max size={0.9\linewidth}{0.9\paperheight}}{output_11_2.png}
    \end{center}
    { \hspace*{\fill} \\}
    
    \begin{Verbatim}[commandchars=\\\{\}]
             precision    recall  f1-score   support

     setosa       1.00      1.00      1.00        10
 versicolor       1.00      0.80      0.89        10
  virginica       0.83      1.00      0.91        10

avg / total       0.94      0.93      0.93        30


    \end{Verbatim}

    \begin{center}
    \adjustimage{max size={0.9\linewidth}{0.9\paperheight}}{output_11_4.png}
    \end{center}
    { \hspace*{\fill} \\}
    
    \begin{center}
    \adjustimage{max size={0.9\linewidth}{0.9\paperheight}}{output_11_5.png}
    \end{center}
    { \hspace*{\fill} \\}
    
    \begin{center}
    \adjustimage{max size={0.9\linewidth}{0.9\paperheight}}{output_11_6.png}
    \end{center}
    { \hspace*{\fill} \\}
    
    \begin{Verbatim}[commandchars=\\\{\}]
             precision    recall  f1-score   support

     setosa       1.00      1.00      1.00        10
 versicolor       1.00      0.90      0.95        10
  virginica       0.91      1.00      0.95        10

avg / total       0.97      0.97      0.97        30


    \end{Verbatim}

    \begin{center}
    \adjustimage{max size={0.9\linewidth}{0.9\paperheight}}{output_11_8.png}
    \end{center}
    { \hspace*{\fill} \\}
    
    \begin{center}
    \adjustimage{max size={0.9\linewidth}{0.9\paperheight}}{output_11_9.png}
    \end{center}
    { \hspace*{\fill} \\}
    
    \begin{center}
    \adjustimage{max size={0.9\linewidth}{0.9\paperheight}}{output_11_10.png}
    \end{center}
    { \hspace*{\fill} \\}
    
    \begin{Verbatim}[commandchars=\\\{\}]
             precision    recall  f1-score   support

     setosa       1.00      1.00      1.00        10
 versicolor       1.00      0.90      0.95        10
  virginica       0.91      1.00      0.95        10

avg / total       0.97      0.97      0.97        30


    \end{Verbatim}

    \begin{center}
    \adjustimage{max size={0.9\linewidth}{0.9\paperheight}}{output_11_12.png}
    \end{center}
    { \hspace*{\fill} \\}
    
    \begin{center}
    \adjustimage{max size={0.9\linewidth}{0.9\paperheight}}{output_11_13.png}
    \end{center}
    { \hspace*{\fill} \\}
    
    \begin{center}
    \adjustimage{max size={0.9\linewidth}{0.9\paperheight}}{output_11_14.png}
    \end{center}
    { \hspace*{\fill} \\}
    
    \begin{Verbatim}[commandchars=\\\{\}]
             precision    recall  f1-score   support

     setosa       1.00      1.00      1.00        10
 versicolor       0.82      0.90      0.86        10
  virginica       0.89      0.80      0.84        10

avg / total       0.90      0.90      0.90        30


    \end{Verbatim}

    \begin{center}
    \adjustimage{max size={0.9\linewidth}{0.9\paperheight}}{output_11_16.png}
    \end{center}
    { \hspace*{\fill} \\}
    
    \begin{center}
    \adjustimage{max size={0.9\linewidth}{0.9\paperheight}}{output_11_17.png}
    \end{center}
    { \hspace*{\fill} \\}
    
    \begin{center}
    \adjustimage{max size={0.9\linewidth}{0.9\paperheight}}{output_11_18.png}
    \end{center}
    { \hspace*{\fill} \\}
    
    \begin{Verbatim}[commandchars=\\\{\}]
             precision    recall  f1-score   support

     setosa       1.00      1.00      1.00        10
 versicolor       1.00      0.90      0.95        10
  virginica       0.91      1.00      0.95        10

avg / total       0.97      0.97      0.97        30


    \end{Verbatim}
Comment on how well you think your classifier will work on a secret test set (e.g. Kaggle's test set):

I don't think my classifier works very well. This is because I know somewhere in the code I couldnt seperate the different groups of flower petals from the train and test data. So the classifier is classifying the entire dataset. Also there isnt much difference in my matrixes and classification_reports. Initially I couldnt figure out how to seperate into groups until I found a kaggle competition online that helped me out with the iris set.

    % Add a bibliography block to the postdoc
    
    
    
    \end{document}
